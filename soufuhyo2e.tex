%#!platex
%%%%%%%%%%%%%%%%%%%%%% end of soufuhyo.sty %%%%%%%%%%%%%%%%%%%%%%%%%%%%%
%%%%%%%%%%%%%%%%%%%%%%%%%%%%%%%%%%%%%%%%%%%%%%%%%%%%%%%%%%%%%%%%%%%%%
%
% 土木学会論文集送付票作成 サンプルソース
%
%%%%%%%%%%%%%%%%%%%%%%%%%%%%%%%%%%%%%%%%%%%%%%%%%%%%%%%%%%%%%%%%%%%%%
%
% for LaTeX2e users

\documentclass[dvipdfmx,11pt]{jarticle}
\usepackage{fancybox,soufuhyo}
 \renewcommand{\baselinestretch}{1.1037}
 \renewcommand{\arraystretch}{1.1037}

\begin{document}


% 題目

\title{\Large 土木工学における土木学会論文集の役割とその版下原稿提出への
移行および電子出版への将来見通し} % 日本語題目

\etitle{\large Future of Publication of JSCE Journals and a New
Form of Publishing with Multi Media} % 英語題目

% 著者データ
%   \author[会員]{姓名}{LAST NAME, First name}{勤務先・職名}
%  会員欄
%         正会員  \author{姓名}{LAST NAME, ....
%   フェロー会員   \author[1]{姓名}{LAST NAME, ....
%       学生会員   \author[2]{姓名}{LAST NAME, ....
%         非会員  \author[3]{姓名}{LAST NAME, ....k
%
%  第一著者のみ,最後に会員番号が必要
\author{建設太郎}{KENSETSU, Taro}{建設大学工学部環境土木工学科・教授}{1234567}

%  第二著者
\authorA[3]{土木花子}{DOBOKU, Hanako}{建設コンサルタント・主任技師}

%  第三著者
\authorB[1]{地球 栄}{CHIKYU, Sakae}{木土理化学研究所・主任}

%  第四著者
\authorC[2]{環境三五郎}{KANKYO, Sangoro}{建設大学工学部土木工学科}

%  第五著者 いないときはコメントアウト
  \authorD{}{}{}

%  第六著者
  \authorE{}{}{}

% 投稿区分: 論文=1, 報告=2, ノート=3, 討議=4, 研究展望=5

\kubun{2}

% 投稿部門: I から VII までを算用数字で

\bumon{5}

%%%%%%%%%%%%%%%%%%%%%%%%%%%%%%%%%%%%%%%%%%%%%%%%%%%%%%%%%%%%%%%%%%%%%%
% 以下の項目で該当しないものは,ブランクにせず,コメントアウトします %
%%%%%%%%%%%%%%%%%%%%%%%%%%%%%%%%%%%%%%%%%%%%%%%%%%%%%%%%%%%%%%%%%%%%%%

% 複数の部門に関係し,査読を上の部門以外にも希望する場合に

\fukubumon{3}

% 過去の発表の経緯 [0]は審査無し,[1]は審査有りです.

\ikisatua[0]{ここには過去の経緯を書きます.a}
\ikisatub[1]{ここには過去の経緯を書きます.b}
\ikisatuc[0]{ここには過去の経緯を書きます.c}
%\ikisatud[1]{ここには過去の経緯を書きます.d}
%\ikisatue[0]{ここには過去の経緯を書きます.e}

% この論文がかつて返却になったものの場合
%
% 返却された論文の題目

\returntitle{この論文の元となったところの返却された論文題目です}

% そのときの部門

\returnbumon{5}

% 投稿区分:論文=1, 報告=2, ノート=3

\returnkubun{2}

% 投稿した時期の年と月を \returndate{西暦年}{月} で
% ^^^^^^^^

\returndate{1993}{9}

% 投稿した期の論文番号 (a - b) \returnno{a}{b} で
% ^^^^^^^^

\returnno{1}{999}

%%%%%%%%%%%%%%%%%%%%%%%%%%%%%%%%%%%%%%%%%%%%%%%%%%%%%%%%%%%%%%%%%%%%%%
% 以下の項目はすべて記入する %%%%%%%%%%%%%%%%%%%%%%%%%%%%%%%%%%%%%%%%%
%%%%%%%%%%%%%%%%%%%%%%%%%%%%%%%%%%%%%%%%%%%%%%%%%%%%%%%%%%%%%%%%%%%%%%
% 全頁数を算用数字で

\totalpage{8}

% カラー頁の数

\colorpage{2}

% 連絡者の住所を書きます. \par が許されます.ただし 2 行以内に.

\address{〒 160 東京都新宿区四谷一丁目無番地\par
 建設大学工学部土木工学科\par
 建設太郎}

% 電話番号: 内線がある時には \telephone[内線]{代表電話}

\telephone[2468]{022-222-1800}

% FAX 番号

\fax{022-268-3689}

% Email アドレス

\email{hoge@achara.domain.univ.ac.jp}

%%%%%%%%%%%%%%%%%%%%%%%%%%%%%%%%%%%%%%%%%%%%%%%%%%%%%%%%%%%%%%%%%%%%%
% ここまでで STOP ! これ以下も手を入れないこと.
\doit
\end{document}
%%%%%%%%%%%%%%%%%%%%%%%%%%%%%%%%%%%%%%%%%%%%%%%%%%%%%%%%%%%%%%%%%%%%%









