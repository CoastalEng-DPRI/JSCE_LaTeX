%%#!platex
%%%%%%%%%%%%%%%%%%%%%%%%%%%%%%%%%%%%%%%%%%%
% bibtex を使って参考文献リストを作ること
%%%%%%%%%%%%%%%%%%%%%%%%%%%%%%%%%%%%%%%%%%%
\documentclass[dvipdfmx]{jsce}
\finalversion % ヘッダが不要らしい
\title{\BibTeX\ の使用例}
\endtitle{USAGE of \BibTeX}
\author{論文集 編集委員会\thanks{正会員 工博 土木大学教授 工学部
土木工学科(〒160-0001東京都新宿区四谷一丁目
無番地)\email{your_name@foo.ac.jp}}}
\endauthor{Editorial COMMITTEE}
\abstract{\BibTeX\ の使用例を示す.}
\keywords{\BibTeX}
\endabstract{% Yes blank line
Example of the bibliography using \BibTeX.}
\receivedate{2012. 12. 11}
\def\BibTeX{{\rm B\kern-.05em{\sc i\kern-.025em b}\kern-.08em
    T\kern-.1667em\lower.7ex\hbox{E}\kern-.125emX}}
\def\JBibTeX{\leavevmode\lower .6ex\hbox{J}\kern-0.15em\BibTeX}
\begin{document}

\maketitle

\section{参考文献の引用とリスト}

参考文献は出現順に番号を振り,その引用箇所で
このように\cite{Asaro,antmanrod}上付き
右括弧付き数字\cite{hillbook}で指示します.

\begin{enumerate}
\item まず,ソースファイルの参考文献リスト部分を
\begin{verbatim}
   \bibliographystyle{jsce}
   \bibliography{bibtexdb}
\end{verbatim}
とします.通常の`{\tt thebibliography}'環境は使いません.
\item ファイル`bibtexdb.bib'は\BibTeX\ 形式の
データベースです.普段からメンテナンスしておくといつも
同じファイルが使えて便利です.
中身を見ていただければ作り方は簡単にわかると思います.
\lastpagecontrol[1cm]{6.5cm}
\item 一度コンパイルします.
\begin{verbatim}
   platex bibtex-j
\end{verbatim}
\item 次に\JBibTeX\ で,データベースから該当する
文献を取り出して成形します.
\begin{verbatim}
   jbibtex bibtex-j
\end{verbatim}
この操作で文献リストファイル`bibtex-j.bbl'ができあがります.
中身を見ていただければ,普段`{\tt thebibliography}'環境に
並べているものができていることがわかります.
\item 再度コンパイルします.
\begin{verbatim}
   platex bibtex-j
\end{verbatim}
\item おしまい.
\item {\bf 注意1:} 土木学会のクラスファイル(スタイルファイル)で
定義した \verb+\CITE+で引用している場合には,その
引数の二つの文献しかリストされませんので,\verb+\CITE+を使わない
ようにしてください.
\item {\bf 注意2:} もしソースファイルを
学会等に提出する場合は,生成された`bibtex-j.bbl'も
忘れずに提出してください.
\end{enumerate}

\bibliographystyle{jsce}
\bibliography{bibtexdb}

\lastpagesettings

\end{document}
